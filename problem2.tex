\section{Scheduling}
\subsection{}
	The problem can be formulated as follows.
    \\
    \\
    Sets:
    \begin{itemize}
      \item $E$: the set of engines
      \item $T$: the set of engine types
    \end{itemize}
    Parameters:
    \begin{itemize}
      \item $s_{ij}$:  the switchover time from engine $i \in E$ to engine $j \in E$.
        \\ If $i=0 $ then this is actually the setup time for engine $i \in E$.
      \item $p_{i}$:   the processing time for engine $i \in E$
      \item $t_{i}$:   the type of engine $i \in E$
    \end{itemize}
    Decision variables:
    \begin{itemize}
      \item $x_{ij}$: yes or no on whether to do the switchover (or setup) of
        engine $i \in E$ to engine $j \in E$
      \item $v_{i}$: the order in which engine $i \in E$ is visited
    \end{itemize}

    Compact formulation capturing objective function and constraints:
    \begin{center}
      \begin{align*}
        \min\quad & \sum_{i \in E} \sum_{j \in E} x_{ij}(s_{ij} + p_{j})
        \\
        \text{s.t.}\quad
        & \sum_{i \in E} x_{ij} = 1,\; \quad \forall j \in E
        \\
        & \sum_{j \in E} x_{ij} = 1,\; \quad \forall i \in E
        \\
        & v_{i} - v_{j} + \abs{E} x_{ij} \leq \abs{E} - 1,\; \forall i,j \in E, i \neq j
        \\
        & x_{i,i} = 0,\; \quad \forall i \in E
        \\
        & s_{i,j} \geq 0,\; \quad \forall i,j \in E
        \\
        & p_{i} \geq 0,\; \quad \forall i \in E
        \\
        & t_{i} \in T,\; \quad \forall i \in E
        \\
        & x_{ij} \in \{0,1\},\; \forall i,j \in E, i \neq j
        \\
        & 1 \leq v_{i} \leq \abs{E},\; \forall i \in E
      \end{align*}
    \end{center}
\subsection{}
An AMPL model has been created to solve this linear program. It is included as \cref{lst:problem2mod}. \Cref{lst:problem2dat} contains the relevant data used for the model and the output of the model is contained in
\cref{lst:problem2out}. The python code to run the model and generate output can be found in \cref{append:p2}.
The output shows that the optimal production sequence is
\[5 \rightarrow 2 \rightarrow 1 \rightarrow 4 \rightarrow 3\]
with a total production time (makespan) of 64 hours. The 0 shown in \cref{lst:problem2out} acts as a source/sink node
to assist with solving the model as well as ensuring setup time for engine 3 is included in the objective value.
\begin{tcolorbox}
    \lstinputlisting[numbers=left, numberstyle=\tiny,
        stepnumber=2, numbersep=5pt,
        basicstyle=\small\fontencoding{T1}\selectfont, label={lst:problem2mod},
        caption={An AMPL model for problem 2.}]{problem2_python/problem2.mod}
\end{tcolorbox}
\begin{tcolorbox}
    \lstinputlisting[numbers=left, numberstyle=\tiny,
        stepnumber=2, numbersep=5pt,
        basicstyle=\small\fontencoding{T1}\selectfont, label={lst:problem2dat},
        caption={The AMPL data for problem 2.}]{problem2_python/problem2.dat}
\end{tcolorbox}
\begin{tcolorbox}
    \lstinputlisting[numbers=left, numberstyle=\tiny,
        stepnumber=2, numbersep=5pt,
        basicstyle=\small\fontencoding{T1}\selectfont, label={lst:problem2out},
        caption={The AMPL output for problem 2.}]{ampl/problem2.amplout}
\end{tcolorbox}
\subsection{}
A Gantt chart for the optimal production sequence has been included in \cref{fig:problem2_3}. It was generated
using the Gemini 3.0 Pro model with the following prompt:
\begin{quote}
Please review the existing python script that generates production schedules. I would like to modify this script to generate 3 independent Gantt charts utilizing the UF Style guide. Let's generate a chart for the optimal setup time from the AMPL model, a greedy algorithm that selects to shortest setup time, using a random selection if there is a tie, and running all commercial engines first followed by military engines. Be sure the Gantt charts display both running time and setup time. Use two different colors to represent the commercial and military engines. Highlight the total setup time at the top of the chart. The output of each chart should be a PDF image.

Make sure each engine is on its own line in the Gantt chart. Make sure each engine in the Gantt chart lists both its setup time and run time in production.
\end{quote}
\begin{figure}[H]
    \centering
    \includegraphics[width=\textwidth]{images/problem2_optimal_gantt.pdf}
    \caption{Optimal Gantt Chart}
    \label{fig:problem2_3}
\end{figure}
\subsection{}
Gantt charts for the greedy (shortest setup time) and commercial first production schedule have been
included as \cref{fig:problem2_4a} and \cref{fig:problem2_4b} respectively. One can see that the
optimal schedule results in a total setup time of 23 hours. The greedy, shortest time first schedule results in a setup
time of 24 hours. The commercial first implementation has a setup time of 26 hours. The naive commercial first approach
results in the least efficient use of time, while the greedy implementation finds a good but not optimal solution.
Using Linear Programming techniques results in an hour savings over the production cycle, which when expanded over multiple
cycles would represent a significant time savings.
\begin{figure}[H]
    \centering
    \includegraphics[width=\textwidth]{images/problem2_greedy_gantt.pdf}
    \caption{Greedy Gantt Chart}
    \label{fig:problem2_4a}
\end{figure}
\begin{figure}[H]
    \centering
    \includegraphics[width=\textwidth]{images/problem2_commercial_first_gantt.pdf}
    \caption{Commercial First Gantt Chart}
    \label{fig:problem2_4b}
\end{figure}
