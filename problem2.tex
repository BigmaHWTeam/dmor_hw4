\section{Scheduling}
\subsection{}
	The problem can be formulated as follows:
    Sets:
    \begin{itemize}
      \item $E$: the set of engines
      \item $T$: the set of engine types
    \end{itemize}
    Parameters:
    \begin{itemize}
      \item $s_{ij}$:  the switchover time from engine $i \in
        E to engine j \in E$. If $i=0 $ then this is the setup time for engine.
      \item $p_{i}$:   the processing time for engine $i \in E$
      \item $t_{i}$:   the type of engine $i \in E$
    \end{itemize}
    Decision variables:
    \begin{itemize}
      \item $x_{ij}$: yes or no on whether to do the switchover (or setup) of 
        engine $i \in E$ to engine $j \in E$
    \end{itemize}

    Compact formulation capturing objective function and constraints:
    \begin{center}
      \begin{align*}
        \min\quad & \sum_{i \in E} \sum_{j \in E} x_{ij}(s_{ij} + p_{j})
        \\
        \text{s.t.}\quad
        & \sum_{j \in e} x_{ij} = 1,\; \quad \forall i \in \{0...|E-1|\}
        \\
        & \sum_{i \in e} x_{ij} = 1,\; \quad \forall j \in \{1...|E|\}
        \\
        & x_{ij} \in \{0,1\},\; \forall i,j \in E, i \neq j
      \end{align*}
    \end{center}
\subsection{}
An AMPL model has been created to solve this linear program. It is included as listing \ref{lst:problem2mod}. Listing 
\ref{lst:problem2dat} contains the relevant data used for the model and the output of the model is contained in 
listing \ref{lst:problem2out}. The python code to run the model and generate output can be found in \ref{append:p2}.
\begin{tcolorbox}
    \lstinputlisting[numbers=left, numberstyle=\tiny,
        stepnumber=2, numbersep=5pt,
        basicstyle=\small\fontencoding{T1}\selectfont, label={lst:problem2mod},
        caption={An AMPL model for problem 2.}]{problem2_python/problem2.mod}
\end{tcolorbox}
\begin{tcolorbox}
    \lstinputlisting[numbers=left, numberstyle=\tiny,
        stepnumber=2, numbersep=5pt,
        basicstyle=\small\fontencoding{T1}\selectfont, label={lst:problem2dat},
        caption={The AMPL data for problem 2.}]{problem2_python/problem2.dat}
\end{tcolorbox}
\begin{tcolorbox}
    \lstinputlisting[numbers=left, numberstyle=\tiny,
        stepnumber=2, numbersep=5pt,
        basicstyle=\small\fontencoding{T1}\selectfont, label={lst:problem2out},
        caption={The AMPL output for problem 2.}]{ampl/problem2.amplout}
\end{tcolorbox}
\subsection{}
A Gantt chart for the optimal production sequence has been included in Figure \ref{fig:problem2_3}
\begin{figure}[H]
    \centering
    \includegraphics[width=\textwidth]{images/problem2_optimal_gantt.pdf}
    \caption{Optimal Gantt Chart}
    \label{fig:problem2_3}
\end{figure}
\subsection{}
Gantt charts for the greedy (shortest setup time) and commercial first production schedule have been
included as figures \ref{fig:problem2_4a} and \ref{fig:problem2_4b} respectively.
\begin{figure}[H]
    \centering
    \includegraphics[width=\textwidth]{images/problem2_optimal_gantt.pdf}
    \caption{Greedy Gantt Chart}
    \label{fig:problem2_4a}
\end{figure}
\begin{figure}[H]
    \centering
    \includegraphics[width=\textwidth]{images/problem2_commercial_first_gantt.pdf}
    \caption{Commercial First Gantt Chart}
    \label{fig:problem2_4b}
\end{figure}
