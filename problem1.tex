\section{}
\subsection{}
\begin{table}[h!]
    \centering
    \begin{tabular}{lccc}
        \toprule
        \textbf{Part} & \textbf{Quantity} & \textbf{Prepreg/part (kg)} & \textbf{Prepreg/assembly (kg)}\\
        \midrule
        Fuselage panel & 15 & 30 & 450\\
        Wing skin & 20 & 20 & 400 \\
        Tail cone section & 15 & 10 & 150 \\
        \bottomrule
        Total & & & 1000
    \end{tabular}
    \caption{Prepreg requirements for aerospace parts.}
    \label{tab:prepreg_req}
\end{table}
%This section is for the typeset model
\subsection{}

\subsection{}
An ampl model has been created to solve this linear program. It is included as listing \ref{lst:problem1mod}. Listing 
\ref{lst:problem1dat} contains the relevant data used for the model and the output of the model is contained in 
\ref{lst:problem1out}. The python code to run the model and generate output can be found in \ref{append:p1}.
\begin{tcolorbox}
    \lstinputlisting[numbers=left, numberstyle=\tiny,
        stepnumber=2, numbersep=5pt,
        basicstyle=\small\fontencoding{T1}\selectfont, label={lst:problem1mod},
        caption={An AMPL model for problem 1.}]{problem1_python/problem1.mod}
\end{tcolorbox}
\begin{tcolorbox}
    \lstinputlisting[numbers=left, numberstyle=\tiny,
        stepnumber=2, numbersep=5pt,
        basicstyle=\small\fontencoding{T1}\selectfont, label={lst:problem1dat},
        caption={The ampl data for problem 1.}]{problem1_python/problem1.dat}
\end{tcolorbox}
\begin{tcolorbox}
    \lstinputlisting[numbers=left, numberstyle=\tiny,
        stepnumber=2, numbersep=5pt,
        basicstyle=\small\fontencoding{T1}\selectfont, label={lst:problem1out},
        caption={The ampl data for problem 1.}]{ampl/problem1.amplout}
\end{tcolorbox}
