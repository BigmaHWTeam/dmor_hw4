\section{Production Planning}
\subsection{}
The integer problem can be formulated
\begin{center}
  \begin{align*}
    \text{maximize } & (30051x_1+22525x_2+25811x_3) \\
    s.t.\quad        & 3x_1+2x_2+2.5x_3\le 600   \\
                     & 0.5x_1+x_2+0.3x_3\le 200  \\
                     & x_1,x_2,x_3 \in \{0,\mathbb{Z}^+\} \\
  \end{align*}
\end{center}
\subsection{}
An integer AMPL model has been created for problem 4 and is included as \cref{lst:problem4integer}.
The data file used for this model is included as \cref{lst:problem4dat}. It should be noted that
the data file is used consistently across all AMPL implementations for problem 4. The output of the integer
model can be found in \cref{lst:problem4integerout}
\begin{tcolorbox}
		      \lstinputlisting[numbers=left, numberstyle=\tiny,
			      stepnumber=2, numbersep=5pt,
			      basicstyle=\small\fontencoding{T1}\selectfont, label={lst:problem4integer},
			      caption={An integer AMPL model for problem 4.}]{problem4_python/integer.mod}
\end{tcolorbox}
\begin{tcolorbox}
  \lstinputlisting[numbers=left, numberstyle=\tiny,
    stepnumber=2, numbersep=5pt,
    basicstyle=\small\fontencoding{T1}\selectfont, label={lst:problem4dat},
    caption={An AMPL data file for problem 4.}]{problem4_python/problem4.dat}
\end{tcolorbox}
\begin{tcolorbox}
  \lstinputlisting[numbers=left, numberstyle=\tiny,
    stepnumber=2, numbersep=5pt,
    basicstyle=\small\fontencoding{T1}\selectfont, label={lst:problem4integerout},
    caption={An AMPL output for the integer model}]{ampl/integer.amplout}
\end{tcolorbox}

\subsection{}
The linear relaxation of the model is included as \cref{lst:problem4relaxation}. The same data file for the
integer model was used. The output is included as \cref{lst:problem4relaxationout}.
\begin{tcolorbox}
		      \lstinputlisting[numbers=left, numberstyle=\tiny,
			      stepnumber=2, numbersep=5pt,
			      basicstyle=\small\fontencoding{T1}\selectfont, label={lst:problem4relaxation},
			      caption={An integer relaxation AMPL model for problem 4.}]{problem4_python/relaxation.mod}
\end{tcolorbox}
\begin{tcolorbox}
  \lstinputlisting[numbers=left, numberstyle=\tiny,
    stepnumber=2, numbersep=5pt,
    basicstyle=\small\fontencoding{T1}\selectfont, label={lst:problem4relaxationout},
    caption={The AMPL output for the integer relaxation model}]{ampl/relaxation.amplout}
\end{tcolorbox}
\subsection{}
The binary search tree for the Branch and Bound search is included as \cref{fig:problem4}. Node 32 was
identified as the optimal solution via the Branch and Bound search. The output for node 32's model is included as
\cref{lst:problem4node32}. The optimal production schedule for the Branch and Bound method match the optimal
solution identified by the integer AMPL model in \cref{lst:problem4integerout}. The model file and output for
each node has been included in \cref{append:p4nodes} \nameref{append:p4nodes}.
\begin{tcolorbox}
  \lstinputlisting[numbers=left, numberstyle=\tiny,
    stepnumber=2, numbersep=5pt,
    basicstyle=\small\fontencoding{T1}\selectfont, label={lst:problem4node32},
    caption={The AMPL output for the optimal node.}]{ampl/branchbound/node32.amplout}
\end{tcolorbox}
\begin{figure}[H]
  \centering
  \includegraphics[width=\textwidth]{images/binary_search_tree}
  \caption{Visualization of Branch and Bound search tree}
  \label{fig:problem4}
\end{figure}
