\section{Emergency Crew Location}
\subsection{}
This problem was solved using the MCFP model provided. The data file was constructed to represent the arcs 
between the nodes and is included as listsing \ref{lst:problem31dat}. The output of the model is included as
listing \ref{lst:problem31out}.
\begin{tcolorbox}
  \lstinputlisting[numbers=left, numberstyle=\tiny,
    stepnumber=2, numbersep=5pt,
    basicstyle=\small\fontencoding{T1}\selectfont, label={lst:problem31dat},
    caption={An AMPL data file for problem 3.}]{problem3_python/MCFP_3_1.dat}
\end{tcolorbox}
\begin{tcolorbox}
  \lstinputlisting[numbers=left, numberstyle=\tiny,
    stepnumber=2, numbersep=5pt,
    basicstyle=\small\fontencoding{T1}\selectfont, label={lst:problem31out},
    caption={An AMPL output for the Problem 3.1}]{ampl/problem3_1.amplout}
\end{tcolorbox}
\subsection{}
This problem was solved using a modified MCFP model included as \ref{lst:problem32mod}. The data file was constructed to represent the arcs 
between the nodes and is included as listsing \ref{lst:problem31dat}. The output of the model is included as
listing \ref{lst:problem32out}.
\begin{tcolorbox}
  \lstinputlisting[numbers=left, numberstyle=\tiny,
    stepnumber=2, numbersep=5pt,
    basicstyle=\small\fontencoding{T1}\selectfont, label={lst:problem32mod},
    caption={The modified MCFP AMPL model for problem 3.2}]{problem3_python/MCFP_3_2.dat}
\end{tcolorbox}
\begin{tcolorbox}
  \lstinputlisting[numbers=left, numberstyle=\tiny,
    stepnumber=2, numbersep=5pt,
    basicstyle=\small\fontencoding{T1}\selectfont, label={lst:problem32dat},
    caption={An AMPL data file for problem 3.2}]{problem3_python/MCFP_3_2.dat}
\end{tcolorbox}
\begin{tcolorbox}
  \lstinputlisting[numbers=left, numberstyle=\tiny,
    stepnumber=2, numbersep=5pt,
    basicstyle=\small\fontencoding{T1}\selectfont, label={lst:problem32out},
    caption={An AMPL output for the Problem 3.2}]{ampl/problem3_2.amplout}
\end{tcolorbox}
