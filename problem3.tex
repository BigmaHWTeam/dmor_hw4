\section{Emergency Crew Location}
\subsection{}
In this problem, we can represent the locations as nodes in a directed graph.
Each power station or candidate location for an emergency crew is located at a node.
For the case where power stations or candidate locations are located between intersections,
we added an additional "prime" node to represent this halfway location.
You will see these prime nodes in the data file are indicaed with a suffix of "p".
The arcs between the nodes represent the possible routes that crews can take to reach
different locations with each arc having a cost representing travel time between locations.
The objective is to minimize the total travel cost while ensuring that all areas are covered by at least one crew.

The problem can be formulated as follows.
    \\
    \\
    Sets:
    \begin{itemize}
      \item $N$: the set of nodes (power stations, candidate locations, intersections)
      \item $A$: the set of arcs (roads) between nodes
    \end{itemize}
    Parameters:
    \begin{itemize}
      \item $c_{a}$:  cost of traversing road (arc) $a \in A$
      \item $i_{a}$:  tail node of arc $a \in A$
      \item $j_{a}$:  head node of arc $a \in A$
      \item $b_{n}$:  supply (number of power stations serviced by crew) or demand (number of power stations) at node $n \in N$
      \item $lb_{a}$: lower bound on flow for arc $a \in A$
      \item $ub_{a}$: upper bound on flow for arc $a \in A$
    \end{itemize}
    Decision variables:
    \begin{itemize}
      \item $x_{a}$: flow on arc $a \in A$
    \end{itemize}

    Compact formulation capturing objective function and constraints:
    \begin{center}
      \begin{align*}
        \min\quad & \sum_{a \in A} c_{a}*x_{a}
        \\
        \text{s.t.}\quad
        & \sum_{a \in A: i_{a} = n}x_{a} - \sum_{a \in A: j_{a} = n}x_{a} = b_{n},\; \quad \forall n \in N
        \\
        & x_{a} \geq lb_{a},\; \quad \forall a \in A
        \\
        & x_{a} \leq ub_{a},\; \quad \forall a \in A
        \\
        & i_{a} \in N,\; \quad \forall a \in A
        \\
        & j_{a} \in N,\; \quad \forall a \in A
      \end{align*}
    \end{center}
This problem was solved using the MCFP model provided. The data file was constructed to represent the arcs 
between the nodes and is included as listsing \ref{lst:problem31dat}. The output of the model is included as
listing \ref{lst:problem31out}. The output indicates that the crew at Node 1 should repair
the power stations at Nodes 3p, 5, and 6p, while the crew at Node 18 should repair the
power stations at Nodes 13, 23p, and 24. The total travel cost for the crew at Node 1 is 16.5,
while the total travel cost for the crew at Node 18 is 30.5. The overall total travel cost is 47.0.
\begin{tcolorbox}
  \lstinputlisting[numbers=left, numberstyle=\tiny,
    stepnumber=2, numbersep=5pt,
    basicstyle=\small\fontencoding{T1}\selectfont, label={lst:problem31out},
    caption={An AMPL output for the Problem 3.1}]{ampl/problem3_1.amplout}
\end{tcolorbox}

\subsection{}
The problem can be formulated as follows.
    \\
    \\
    Sets:
    \begin{itemize}
      \item $N$: the set of nodes (power stations, candidate locations, intersections)
      \item $A$: the set of arcs (roads) between nodes
      \item $P$: the set of power stations
    \end{itemize}
    Parameters:
    \begin{itemize}
      \item $c_{a}$:  cost of traversing road (arc) $a \in A$
      \item $i_{a}$:  tail node of arc $a \in A$
      \item $j_{a}$:  head node of arc $a \in A$
      \item $lb_{a}$: lower bound on flow for arc $a \in A$
      \item $ub_{a}$: upper bound on flow for arc $a \in A$
      \item $z$:      number of crews available
    \end{itemize}
    Decision variables:
    \begin{itemize}
      \item $x_{a}$: flow on arc $a \in A$
      \item $s_{n}$: outflow from node $n \in N$ where a repair crew is located
    \end{itemize}

    Compact formulation capturing objective function and constraints:
    \begin{center}
      \begin{align*}
        \min\quad & \sum_{a \in A} c_{a}*x_{a}
        \\
        \text{s.t.}\quad
        & \sum_{a \in A: i_{a} = n}x_{a} - \sum_{a \in A: j_{a} = n}x_{a} = -1,\; \quad \forall n \in N: n \in P
        \\
        & \sum_{a \in A: i_{a} = n}x_{a} - \sum_{a \in A: j_{a} = n}x_{a} = s_{n},\; \quad \forall n \in N: n \not\in P
        \\
        & x_{a} \geq lb_{a},\; \quad \forall a \in A
        \\
        & x_{a} \leq ub_{a},\; \quad \forall a \in A
        \\
        & i_{a} \in N,\; \quad \forall a \in A
        \\
        & j_{a} \in N,\; \quad \forall a \in A
        \\
        & s_{n} \geq 0,\; \quad \forall n \in N
      \end{align*}
    \end{center}
This problem was solved using a modified MCFP model included as \ref{lst:problem32mod}. The data file was constructed to represent the arcs 
between the nodes and is included in \ref{append:p3dat} as listsing \ref{lst:appendp32dat}. The output of the model is included as
listing \ref{lst:problem32out}.
\begin{tcolorbox}
  \lstinputlisting[numbers=left, numberstyle=\tiny,
    stepnumber=2, numbersep=5pt,
    basicstyle=\small\fontencoding{T1}\selectfont, label={lst:problem32mod},
    caption={The modified MCFP AMPL model for problem 3.2}]{problem3_python/MCFP_3_2.mod}
\end{tcolorbox}
\begin{tcolorbox}
  \lstinputlisting[numbers=left, numberstyle=\tiny,
    stepnumber=2, numbersep=5pt,
    basicstyle=\small\fontencoding{T1}\selectfont, label={lst:problem32out},
    caption={An AMPL output for the Problem 3.2}]{ampl/problem3_2.amplout}
\end{tcolorbox}
